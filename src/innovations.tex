It's useful for us to first briefly present some of our ``greatest
hits'' in robot engineering and innovation this season, and then go into
further depth on our specific implementation in later sections.

\subsection{Drivebase}

Switching from tank drive to swerve drive was a big decision, and meant
we had to do a lot of things differently. First, we couldn't draw on any
previous designs or code, and basically had to start fresh with a brand
new drivebase design. Luckily, our Rev Robotics MAXSwerve
modules~\cite{revswervemodule} had ample documentation and integration
with WPILib, so we were able to build our drivebase prior to kickoff,
and begin writing a library for controlling it early into the season.

% TODO: add an image of the swerve drivebase


\subsection{Vision}

We vastly expanded our computer vision capabilities this year. In
2023, we had a Microsoft webcam which streamed a low-resolution camera
feed to the driver dashboard, and had no practical application for even
basic vision tasks, such as identifying visual fiducials like april
tags. This year, we got our hands on two Limelight v3s~\cite{limelight},
which are a full low/no code computer vision solution, with
out-of-the-box april tag detection. It also integrates seamlessly with
our new Coral TPU~\cite{coral}, which provides hardware acceleration for
Limelight to run neural networks such as note classifiers, which enable
the detection of game pieces via machine learning algorithms, previously
unfeasible for us. We also experimented with Microsoft's Kinect system,
which, though originally designed for motion gaming on the Xbox, has
proven to be an extremely powerful depth sensor and motion capture tool,
a reputation proven by its use in research laboratories around the
world~\cite{kinectpaper}. In the end, we made the controversial and
difficult decision to abandon the Kinect in production this year due to
extensive cost-benefit analysis determining that the opportunity cost of
diverting precious build season time to it was too great of a risk.\@


\subsection{Driver Controls}

We also focused on creating an improved driver experience this year. One
of our main issues in previous years was the lack of driver practice and
experience operating the robot and its control interface. This was
partly due to poor time management on our part and partly due to
unintuitively designed controls and user interfaces. One of our main
goals this year was to create a slick and easy driver experience to
reduce as much human error and friction as possible. We implemented
multiple separate drive modes and selectors to allow a customizable
experience. The de facto standard for FRC robot driver dashboard is the
Shuffleboard, included as part of the FRC Game Tools software
distribution from LabView. This piece of archaic technology has an
outdated user interface that can only provide rudimentary data and
visualizations from the robot. This season however, we decided to create
our own proprietary driver dashboard for the first time. This dashboard,
dubbed ``Jankboard,'' has features like collision detection, 3D robot
visualization, and links directly to our other driver features like
speed and drive modes.